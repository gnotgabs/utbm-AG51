\chapter{Travaux dirigés n\textdegree 1 : Le codage arithmétique à clés
publiques}
\section{Rappel : Le système RSA}

La génération de clés RSA se fait en suivant cette méthode :
\begin{enumerate}
  \item choix de 2 grands nombres premiers $p$ et $q$.
  \item Calcul de $m = p \times q$ et $\phi (m)  = (p-1)(q-1)$
  \item choix de $r$ tel que $PGCD(r, \phi (m)) = 1$
  \item calcul de $S \equiv r^{-1} [\phi (m)]$
  \item publication de $(m,r)$ clé publique
  \item le couple $(s,m)$ forme la clé privée
\end{enumerate}

Pour crypter le message $M$, on applique la fonction $C(M) \equiv M^r [m]$
.


Pour décoder le message $M$, on applique la fonction $d(M) \equiv M^s [m]$
.
\section{Exercice 1}
\begin{verbatim}
Debut
  result = false
  Si modulo(nombre,2) == 0
  Alors
    result = true
  Sinon
    Pour i = 3, inférieur à racine de nombre, i+= 2
    Faire
      Si modulo(nombre,i) == 0
      Alors
        result = true
      Fin Si
    Fin Pour
  Fin si
  retourner result
Fin
\end{verbatim}

\subsection{Calcul de la complexité}
$Cout = O(\sqrt{m})$
Si $m = m_0, m_1, ... m_{n-1}$ avec $m \in \{0,1\}$.

$m$ étant fonction de $n$, il nous faut calculer la complexité de $n$.
\begin{align*}
n &= \frac{log(m+1)}{log2}\\
nlog(2) &= log(m+1)\\
log(2^n) &= log(m+1)\\
2^n &= m+1 \Rightarrow m=2^n-1\\
Cout &= O(racine(m)) = O(2^{n/2})
\end{align*}

\section{Exercice 2}
\subsection{Donner l'algorithme}
\begin{verbatim}
Debut
  Pour i=2, inférieur à nombre, i++
  Faire
    Si i^(nombre-1)%nombre == 1
    Alors
      Retourner Vrai
    Fin si
  Fin Pour
  Retourner Faux
Fin
\end{verbatim}

\subsection{Donner la complexité}
Soit $m = (m_0, m_1, ... M_n-1), m \in \{0,1\}$.
$m = \displaystyle{\sum^{n-1}_{i=0}}m_i2^i$
\begin{align*}
a^m &= a^{\displaystyle{\sum^{n-1}_{i=0}}m_i2^i}\\
&= \displaystyle{\prod^{n-1}_{i=0}}(a^{2^i})^{m_i}
\end{align*}
La complexité est donc en $O(n^3)$.

\section{Exercice 3}
\subsection{Montrer que $C=d^{-1}$}
\begin{align*}
C(x) &\equiv x^r[m]\\
d(x) &\equiv x^s[m]
\end{align*}

Remarque : $\forall x, si m = pq, x^{(p-1)(q-1)} \equiv 1[m]$.

Ainsi, $\forall m$, si $m = pq$
\begin{align*}
d(c(x)) &\equiv (x^r[m])^s[m]\\
&\equiv x^{rs}[m]
\end{align*}
or $rs \equiv 1[(p-q)(q-1)]$\\
$\Rightarrow d(c(x)) \equiv x[m]$\\
$\Rightarrow d(c(x)) = x$\\
$\Rightarrow d = c^{-1}$\\

\section{Exercice 4}
\subsection{Questions}
\begin{enumerate}
\item Chiffrer le message \verb?21? avec la clé publique $(103, 143)$\\
Remarque: $21^4 \equiv 1[143]$
\item Sachant que $m = 143 = 11 * 13$, calculer la clé privée associée à la 
clé publique $(103, 143)$
\item déchiffrer le message obtenu
\end{enumerate}

\subsection{Réponses}
\begin{enumerate}
\item
\begin{align*}
C(21) &\equiv 21^{103}[143]\\
      &\equiv 21^{4*25+3}[143]\\
      &\equiv (21^4)^{25} \times 21^3 [143]\\
      &\equiv 21^3[143]\\
      &\equiv 109[143]
\end{align*}
Le message chiffré est donc 109.
\item 
\begin{align*}
rs &\equiv 1[(p-1)(q-1)]\\
s &\equiv r^{-1}[120]\\
  &\equiv 103^{-1}[120]\\
  &\equiv 7[120] \text{ par l'algo d'euclide généralisé}
\end{align*}

la clé privée est donc $(7,143)$.

\item $d(109) \equiv 109^7[143]$
\end{enumerate}
