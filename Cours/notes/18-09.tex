\chapter*{Présentation de l'UV}
\addcontentsline{toc}{chapter}{\numberline{}Présentation de l'UV}

Le barème se présente ainsi :
\begin{itemize}
\item TP 40\%
\item Final 60\%
\end{itemize}
\vspace*{2ex}

\indent Les objectifs de l'UV sont :
\begin{itemize}
  \item Connaitre et maitriser les paradigmes de conception des algos
    \begin{itemize}
      \item[*] paradigme itératif
      \item[*] paradigme diviser pour regner
      \item[*] paradigme de programmation dynamique
      \item[*] paradigmes gloutons
      \item[*] paradigmes de recherche exhaustive
      \item[*] paradigme des algos d'approximation
      \item[*] la complétude NP
      \item[*] algorithmique parallèle (UPC)
      \item[*] algorithmes aléatoires
    \end{itemize}
  \item Comprendre les textes algorithmiques actuels
  \item Étudier les structures de données avancées
\end{itemize}

\chapter{Analyse des algorithmes}
L'analyse des algorithmes, c'est :
\begin{itemize}
  \item estimer les ressources CPU et mémoire nécessaires à son exécution
  \item comparer les algorithmes entre eux
  \item déterminer les algos les plus efficaces
\end{itemize}

La mesure du coût d'un algorithme repose sur le deux facteurs, la mesure du
coût en temps CPU (coût temporel) et la mesure du coût en espace mémoire.

\section{Analyse d'un algorithme : le tri par insertion}

Le principe du tri par insertion est très simple : il repose sur la méthode
\og naturelle \fg que nous utilisons pour le tri. Au fur et à mesure des
itérations dans la liste, nous insérons les éléments à leur place parmis les
éléments précédemment parcourus.

Voici l'algorithme utilisé pour un tableau A donné :
\begin{lstlisting}
Variables :
  i, j, cle, entiers

DEBUT
  POUR j allant de 2 a taille(A)
  REPETER
    cle = A[j]
    //Insertion de la cle dans le sous tableau
    i = j-1
    TANT QUE i >0 ET A[i] > cle
    FAIRE
      A[i+1]  = A[i]
      i=i-1
    FIN TANT QUE
    A[i+1] = cle
  FIN POUR
FIN
\end{lstlisting}
Calculons maintenant la complexité de cet algorithme :
\begin{itemize}
  \item la boucle \verb?POUR? est exécutée \verb?n? fois, \verb?n? étant le
nombre d'éléments du tableau (coût $C_1$)
  \item les lignes 7 et 9 sont exécutées \verb?n-1? fois (coûts $C_2$ et
$C_3$)
  \item la ligne 10 est exécutée $\displaystyle{\sum_{j=2}^n}t_j$ fois (coût
$C_4$)
  \item les lignes 12 et 13 sont exécutées $\displaystyle{
\sum_{j=2}^n}t_j-1$ fois, (coûts $C_5$ et $C_6$)
  \item la ligne 15 est exécutée \verb?n-1? fois (coût $C_7$)
\end{itemize}

On peut donc calculer le coût de l'algorithme par la formule :
$T(n) = nC_1 + (n-1)C_2 + (n-1)C_3 + C_4\displaystyle{\sum_{j=2}^n}t_j +
C_5\displaystyle{\sum_{j=2}^n}(t_j-1)  + C_6\displaystyle{\sum_{j=2}^n}(t_j
-1) + (n-1)C_7$

\subsection{Cas le plus favorable}
Dans ce cas, le tableau est déjà trié et $t_j=1$.

Donc $T(n) = nC_1 + (n-1)C_2 + (n-1)C_3 + (n-1)C_4 + (n-1)C_7 = n(C_1 + C_2
+ C_3 + C_4 + C_7) - (C_2 + C_3 + C_4 + C_7) = a\times n + b$

\subsection{Cas le moins favorable}
$t_j = j$, le tableau est trié à l'envers.
Donc $T(n) = nC_1 + (n-1)C_2 + (n-1)C_3 + (\frac{n(n-1)}{2}-1)C_4 + 
(\frac{n(n-1)}{2})C_5 +(\frac{n(n-1)}{2})C_6 +(n-1)C_7 = a\times n^2 +
b\times n + c$

\subsection{Cas moyen}
$t_j = \frac{j}{2}$

Le calcul est le même que dans le cas le mois favorable et on obtient un
T(n) de la forme $an^2 + bn + c$.

\paragraph{Remarque importante}
Les constantes $C_i$ sont liées à la machine, c'est pour cela que l'on
s'intéresse plutôt à l'ordre de grandeur de $T(n)$ qui est une fonction qui
dépend de $n$.

$T(n) = an^2 + bn + c = O(n^2)$

\section{Deuxième exemple : le tri fusion}

Ce tri utilise le paradigme \textit{Diviser pour régner}.
